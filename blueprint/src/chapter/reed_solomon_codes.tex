\chapter{The Reed-Solomon code}

\begin{definition}[Error-Correcting Code]\label{def:eccode}
    An \emph{error-correcting code} of length \(n\) over an alphabet \(\Sigma\) is a subset \(\code \subseteq \Sigma^n\). The code \(\code\) is called a \emph{linear code} if
    \(\Sigma = \field\) is a finite field and \(\code\) is a subspace of \({\field}^n\).
\end{definition}

\begin{definition}[Reed-Solomon Code]\label{def:rscode}
\lean{polynomialsUpTo,polynomialEvalOn,ReedSolomonCode}
\leanok
    The \emph{Reed-Solomon code} over finite field $\field$, evaluation domain $\evaldomain\subseteq\field$ and degree $d\in\N$ is the set of evaluations (over $\evaldomain$) of univariate polynomials (over $\field$) of degree less than $\degree$:
    \[
        \rscode[\field,\evaldomain,\degree]:=\; \bigl\{\, f : \evaldomain \to \field \;\big|\; \exists \,\hat{f} \,\in \field^{<\degree}[X]\text{ such that } \forall x \in \evaldomain,\; f(x) = \hat{f}(x)\bigr\}.
    \]

    The rate of $\rscode[\field,\evaldomain,\degree]$ is $\rate := \frac{\degree}{\lvert \evaldomain \rvert}$.

    Given a code $\code := \rscode[\field,\evaldomain,\degree]$ and a function $f : \evaldomain \to \field$, we sometimes use $\hat{f} \in \field^{<\degree}[X]$ to denote a nearest polynomial to $f$ on $\evaldomain$ (breaking ties arbitrarily).
\end{definition}

\begin{remark}
Note that the evaluation domain $\evaldomain\subseteq\field$ is a non-empty set.
\end{remark}

\begin{definition}\label{def:list_decodable}
%\lean{}
%\leanok
\uses{def:rscode}
    For a Reed-Solomon code $\code := \rscode[\field,\evaldomain,\degree]$, parameter $\distance \in [0,1]$, 
    and a function $f:\evaldomain\to\field$, let $\listcode(f,\degree,\distance)$ denote the list 
    of codewords in $\code$ whose relative Hamming distance from $f$ is at most $\distance$.
    We say that $\code$ is \emph{$(\distance,\degree)$-list decodable} if 
    \[
    \bigl|\listcode(f,\degree,\distance)\bigr| < |L|
    \quad
    \text{for every function } f.
    \]
\end{definition}
    
    \noindent
    The Johnson bound provides an upper bound on the list size of this Reed-Solomon code:
    
    \begin{theorem}[Johnson bound]\label{thm:johnson_bnd}
    %\lean{}
    %\leanok
    \uses{def:rscode,def:list_decodable}
    The Reed-Solomon code $\rscode[\field,\evaldomain,\degree]$ is $(1-\sqrt{\rate}-\eta,\frac{1}{2\eta\rate})$-list-decodable for every $\eta\in(0,1-\sqrt{\rate})$, where $\rate:=\frac{\degree}{|\evaldomain|}$ is the rate of the code.
    \end{theorem}
    