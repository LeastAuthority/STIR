\chapter{STIR}
\begin{theorem}[STIR Main Theorem]\label{thm:stir}
    Consider the following ingrediants:
    \begin{itemize}
        \item A security parameter $\lambda\in\N$.
        \item A Reed-Solomon code $\rscode[\field,\evaldomain,\degree]$ with $\rate:=\frac{\degree}{|\evaldomain|}$ where $\degree$ is a power of $2$, and $\evaldomain$ is a smooth domain.
        \item A proximity parameter $\distance\in(0,1-1.05\cdot\sqrt{\rate})$.
        \item A folding parameter $k\in\N$ that is power of $2$ with $k\geq 4$.
    \end{itemize}
If $|\field|=\Omega(\frac{\lambda\cdot2^\lambda\cdot\degree^2\cdot{|\evaldomain|}^2}{\log(1/\rate)})$, there is a public-coin IOPP for $\rscode[\field,\evaldomain,\degree]$ with the following parameters:
\begin{itemize}
    \item Round-by-round soundness error $2^{-\lambda}$.
    \item Round complexity: $M:=O(\log_k{\degree})$.
    \item Proof length: $|\evaldomain|+O_k(\log{\degree})$.
    \item Query complexity to the input: $\frac{\lambda}{-\log{(1-\distance)}}$.
    \item Query complexity to the proof strings: $O_k(\log{\degree}+\lambda\cdot\log{\Big(\frac{\log{\degree}}{\log{1/\rate}}\Big)})$.
\end{itemize}
\end{theorem}

\subsection{The STIR Construction}\label{subsec:stir_constr}
Consider the following ingrediants:
\begin{itemize}
    \item a field $\field$,
    \item an iteration count $M\in\N$,
    \item an initial degree parameter $\degree\in\N$ that is a power of $2$,
    \item a folding parameters $k_0,\ldots,k_M\in\N$ that are powers of $2$ with $\degree\geq\prod_{i}k_i$,
    \item evaluation domains $\evaldomain_0,\ldots,\evaldomain_M\subseteq\field$ where $\evaldomain_i$ is a smooth coset of ${\field}^*$ with $|\evaldomain_i|>\frac{\degree}{\prod_{j<i}k^j}$
    \item repetition parameters $t_0,\ldots,t_M\in\N$ where $t_i+1\leq\frac{\degree}{\prod_{j\leq i}k^j}$ for every $i\in\{0,\ldots,M-1\}$,
    \item out of domain repetition parameter $s\in\N$.
\end{itemize}
For every $i\in\{0,\ldots,M\}$, set $\degree_i:=\frac{\degree}{\prod_{j<i}k^j}$. The protocol proceeds as follows.

\begin{itemize}
  \item\textbf{Initial function:} Let $f_0:\evaldomain\rightarrow\field$ be an oracle function. In the honest case, $f_0=\rscode[\field,\evaldomain_0,\degree_0]$ and the prover has access to the polynomial $\hat{f}\in\field^{<\degree_0}[X]$ whose restriction to $\evaldomain_0$ is $f_0$.
  \item \textbf{Initial folding:} The verifier sends $r^{\mathsf{Fold}}\gets\field$
  \item \textbf{Interaction phase loop:} For $i\in\{1,\ldots,M\}$:
  \begin{enumerate}
    \item \textbf{Send folded function:} The prover sends a function $g_i:\evaldomain_i\rightarrow\field$. In the honest case $g_i$ is the evaluation of the polynomial $\hat{g}_i:=\mathsf{PolyFold}(\hat{f}_{i-1},k_{i-1},r^{\mathsf{fold}}_{i-1})$ over $\evaldomain_i$.
    \item \textbf{Out-of-domain samples:} The verifier sends $r_{i,1}^{\mathsf{out}},\ldots,r_{i,s}^{\mathsf{out}}\in\field\setminus\evaldomain_i$
    \item \textbf{Out-of-domain reply:}
    The prover sends field elements
    $\beta_{i,1},\dots,\beta_{i,s}\in\field$. In the honest case,
    $\beta_{i,j}:=\hat{g}_{i}(r_{i,j}^{\mathsf{out}})$.
    \item \textbf{STIR message:} The verifier sends $r^{\mathsf{fold}}_i,r^{\mathsf{shift}}_i\in\field$ and $r_{i,1}^{\mathsf{shift}},\ldots,r_{i,t_{i-1}}^{\mathsf{shift}}\gets\evaldomain^{k_i-1}_{i-1}$
    \item \textbf{Define next polynomial and send hole fills:} The prover sends the oracle message $\mathsf{Fill}_i:=(r^{\mathsf{shift}}_{i,1},\ldots,r^{\mathsf{shift}}_{i,t_{i-1}})\cap\evaldomain_i\rightarrow\field$. In the honest case, the prover defines $\mathcal{G}_i=\{r^{\mathsf{out}}_{i,1},\ldots,r^{\mathsf{out}}_{i,s},r^{\mathsf{shift}}_{i,1},\ldots,r^{\mathsf{shift}}_{i,t_{i-1}}\}$, $\hat{g}'_i:=\mathsf{PolyQuotient}(\hat{g}_i,\mathcal{G}_i)$ and $\mathsf{Fill}_i(r^{\mathsf{shift}}_{i,j}):=\hat{g}'_i(r^{\mathsf{shift}}_{i,j})$ $(\text{ If }r^{\mathsf{shift}}_{i,j}\in\evaldomain_i)$

    Additionally, the honest prover defines the degree-corrected polynomial 
    $\hat{f}_i \in \field^{<\degree}[X]$ as follows:
    \[
        \hat{f}_i 
        :=\degcorr(\degree_i,r_{i}^{\mathsf{comb}},\hat{g}'_i,\degree_i-|\mathcal{G}_i|)
    \]
  The protocol proceeds to the next iteration with $\hat{f}_i$.
  \end{enumerate}   
\end{itemize}

