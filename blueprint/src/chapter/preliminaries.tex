/-
Copyright (c) 2025 ZKLib Contributors. All rights reserved.
Released under Apache 2.0 license as described in the file LICENSE.
Authors: Least Authority
-/

\chapter{Preliminaries}

\begin{definition}[Interactive Oracle Proofs of Proximity (IOPP)]\label{def:iopp}
    A \emph{$k$‑round public‑coin interactive‑oracle proof of proximity
    (IOPP)} for a ternary relation $\mathcal R = \{(x,y,w)\}$ is an
    interactive protocol between a prover $\mathsf P$ and a verifier
    $\mathsf V$ defined as follows.
    
    \begin{itemize}
      \item The prover receives $(x,y,w)$, while the verifier receives $x$
            and oracle access to $y$.
    
      \item For each round $i\in[k]$ the verifier sends a uniformly random
            message $\alpha_i$ to the prover, who responds with a proof
            string $\pi_i$.
    
      \item After $k$ rounds, the verifier may query $y$ and the proof
            strings $\pi_1,\dots,\pi_k$ and finally outputs a decision bit.
    \end{itemize}
    
    Formally, let $\IOP=(\mathsf P,\mathsf V)$ where $\mathsf P$ is an
    interactive algorithm and $\mathsf V$ is an interactive‑oracle algorithm.
    The protocol has \textbf{perfect completeness} and \textbf{soundness
    error} $\beta$ if the following conditions hold.
    
    \paragraph{Perfect completeness.}
    For every $(x,y,w)\in\mathcal R$,
    \[
      \Pr_{\alpha_1,\ldots,\alpha_k}\!\Bigl[
        \mathsf V^{y,\pi_1,\ldots,\pi_k}(x,\alpha_1,\ldots,\alpha_k)=1
        \;\Big|\;
        \pi_1\leftarrow\mathsf P(x,y,w),\;
        \dots,\;
        \pi_k\leftarrow\mathsf P(x,y,w,\alpha_1,\ldots,\alpha_k)
      \Bigr]=1.
    \]
    
    \paragraph{Soundness.}
    For every $(x,y)\notin L(\mathcal R)$ and every (unbounded) malicious
    prover $\widetilde{\mathsf P}$,
    \[
      \Pr_{\alpha_1,\ldots,\alpha_k}\!\Bigl[
        \mathsf V^{y,\pi_1,\ldots,\pi_k}(x,\alpha_1,\ldots,\alpha_k)=1
        \;\Big|\;
        \pi_1\leftarrow\widetilde{\mathsf P}(\alpha_1),\;
        \dots,\;
        \pi_k\leftarrow\widetilde{\mathsf P}
              (x,y,\alpha_1,\ldots,\alpha_k)
      \Bigr]\le \beta(x,y).
    \]
    
    When the soundness error depends only on the input lengths and on the
    proximity $\delta$ of $y$ to the language
    \[
      L_x \;:=\; \{\,y' \mid \exists w,\,(x,y',w)\in\mathcal R\},
    \]
    we write $\beta\bigl(|x|,|y|,\delta\bigr)$, or simply
    $\beta(\delta)$ when $|x|$ and $|y|$ are clear from context.
    \end{definition}
    
\begin{definition}
    Let $k\in \N$ be an integer, $\field$ be a finite field and $\evaldomain \subset \field$ be a subset of $\field$. Then 
    \[
    \evaldomain^k := \{x^k \text{ s.t. } x \in \evaldomain\}
    \]   
\end{definition}

\begin{definition}[Reed-Solomon Code]\label{def:rscode}
\lean{polynomialsUpTo,polynomialEvalOn,ReedSolomonCode}
\leanok
    The \emph{Reed-Solomon code} over finite field $\field$, evaluation domain $\evaldomain\subseteq\field$ and degree $d\in\N$ is the set of evaluations (over $\evaldomain$) of univariate polynomials (over $\field$) of degree less than $\degree$:
    \[
        \rscode[\field,\evaldomain,\degree]:=\; \bigl\{\, f : \evaldomain \to \field \;\big|\; \exists \,\hat{f} \,\in \field^{<\degree}[X]\text{ such that } \forall x \in \evaldomain,\; f(x) = \hat{f}(x)\bigr\}.
    \]

    The rate of $\rscode[\field,\evaldomain,\degree]$ is $\rate := \frac{\degree}{\lvert \evaldomain \rvert}$.

    Given a code $\code := \rscode[\field,\evaldomain,\degree]$ and a function $f : \evaldomain \to \field$, we sometimes use $\hat{f} \in \field^{<\degree}[X]$ to denote a nearest polynomial to $f$ on $\evaldomain$ (breaking ties arbitrarily).
\end{definition}

\begin{remark}
Note that the evaluation domain $\evaldomain\subseteq\field$ is a non-empty set.
\end{remark}

\begin{definition}\label{def:list_decodable}
%\lean{}
%\leanok
\uses{def:rscode}
    For a Reed-Solomon code $\code := \rscode[\field,\evaldomain,\degree]$, parameter $\distance \in [0,1]$, 
    and a function $f:\evaldomain\to\field$, let $\listcode(f,\degree,\distance)$ denote the list 
    of codewords in $\code$ whose relative Hamming distance from $f$ is at most $\distance$.
    We say that $\code$ is \emph{$(\distance,l)$-list decodable} if 
    \[
    \bigl|\listcode(f,\degree,\distance)\bigr| \leq l
    \quad
    \text{for every function } f.
    \]
\end{definition}
    
    \noindent
    The Johnson bound provides an upper bound on the list size of this Reed-Solomon code:
    
    \begin{theorem}[Johnson bound]\label{thm:johnson_bnd}
    %\lean{}
    %\leanok
    \uses{def:rscode,def:list_decodable}
    The Reed-Solomon code $\rscode[\field,\evaldomain,\degree]$ is $(1-\sqrt{\rate}-\eta,\frac{1}{2\eta\rate})$-list-decodable for every $\eta\in(0,1-\sqrt{\rate})$, where $\rate:=\frac{\degree}{|\evaldomain|}$ is the rate of the code.
    \end{theorem}
    